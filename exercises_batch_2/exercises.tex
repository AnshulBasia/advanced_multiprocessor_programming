\documentclass[a4paper,10pt]{article}

\usepackage{amsmath}
\usepackage{amssymb}
\usepackage[usenames,dvipsnames]{color}
\usepackage{comment}
\usepackage[utf8]{inputenc}
\usepackage{listings}

\definecolor{OliveGreen}{cmyk}{0.64,0,0.95,0.40}
\definecolor{Gray}{gray}{0.5}

\lstset{
    language=Java,
    basicstyle=\ttfamily,
    keywordstyle=\color{OliveGreen},
    commentstyle=\color{Gray},
    captionpos=b,
    breaklines=true,
    breakatwhitespace=false,
    showspaces=false,
    showtabs=false,
    numbers=left,
}

\title{VU Advanced Multiprocessor Programming \\
       SS 2013 \\
       Exercises Batch 2}
\author{Jakob Gruber, 0203440}

\begin{document}

\maketitle

\tableofcontents

\pagebreak

\section{Specifications}

Select any 10 of $\{21, 22, 23, 24, 27, 32, 51, 52, 53, 54, 58, 62, 65, 68\}$ 
from \emph{Maurice Herlihy, Nir Shavit: The Art of Multiprocessor Programming. 
Morgan Kaufmann, 2008. Revised 1st Edition, 2012.}

\section{Solutions}

\subsection{Exercise 53}

\emph{The \lstinline|Stack| class provides two methods: \lstinline|push(x)| pushes a value onto the top of the stack, and \lstinline|pop()| removes and returns the most recently pushed value. Prove that the \lstinline|Stack| class has consensus number exactly two.}

\vspace{3mm}

\emph{Consensus object}: provides a method \lstinline|T decide(T value)| which is called by each thread at most once. It is \emph{consistent} (all threads decide the same value) and \emph{valid} (the common decision value is some thread's input).

\emph{Consensus protocol}: a solution to the consensus problem that is wait-free (and therefore also lock-free).

\emph{Consensus number}: A class C solves n-thread consensus if there exists a consensus protocol using any number of objects of class C and any number of atomic registers. The consensus number is the largest n for which class C solves n-thread consensus.

This proof is similar to the FIFO Queue proof in the book, pages 108 to 110. To show that the consensus number of \lstinline|Stack| is at least two, we construct a consensus protocol as in Figure \ref{fig:stackconsensus}.

\begin{figure}
\begin{lstlisting}
class StackConsensus<T> {
    Stack s;
    T[] proposed;
    StackConsensus() {
        s.push(LOSE);
        s.push(WIN);
    }
    T decide(T value) {
        proposed[threadID] = value;
        if (s.pop() == WIN) {
            return proposed[threadID];
        } else {
            return proposed[1 - threadID];
        }
    }
}
\end{lstlisting}
\caption{A stack consensus protocol.}
\label{fig:stackconsensus}
\end{figure}

This protocol is wait-free since \lstinline|Stack| is wait-free and \lstinline|decide()| contains no loops. If each thread returns its own input, both must have popped \lstinline|WIN|, violating the \lstinline|Stack| protocol. Likewise, both threads returning the other's value also violates the protocol. Additionally, the protocol must return one of the proposed values because the winning value is written before \lstinline|WIN| is popped.

We now need to show that \lstinline|Stack| has a consensus number of exactly two. Assume we have a consensus protocol for threads A, B, and C. According to Lemma 5.1.3, there must be a critical state s. Without loss of generality, we assume that A's next move takes to protocol to a 0-valent state, and B's next move leads to a 1-valent state. We also know that these calls must be non-communtative; this implies that they need to be calls on the same object. Next, we know that these calls cannot be made to registers since registers have a consensus number of 1. Therefore, these calls must be made to the same stack object. We can now distinguish between three cases: either both A and B call \lstinline|push()|, both call \lstinline|pop()|, or A calls \lstinline|push()| while B calls \lstinline|pop()|.

Suppose both A and B call \lstinline|pop()|. Let $s^{'}$ be the state if A pops, followed by B; and $s^{''}$ if the pops occur in the opposite order. Since $s^{'}$ is 0-valent while $s^{''}$ is 1-valent, and C cannot distinguish between both states, it is impossible for C to decide the correct value in both states.

Suppose A calls \lstinline|push()| while B calls \lstinline|pop()|. % TODO

\vspace{3mm}


\begin{comment}

Exercise 51: p139

\subsection{Exercise }

\emph{}

\vspace{3mm}

\vspace{3mm}

\begin{align}
S(p) &= \frac{s}{s'} = \frac{2s(p - 1)}{s(p - 1) - 1} \\
     &= \frac{2 \cdot \frac{3}{10} \cdot (p - 1)}{\frac{3}{10} \cdot (p - 1) - 1} 
\end{align}

\begin{figure}
\begin{lstlisting}

\end{lstlisting}
\caption{The Flaky lock used in Exercise 11.}
\label{fig:flaky}
\end{figure}

\end{comment}

\end{document}

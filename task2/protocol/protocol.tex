\documentclass[a4paper,10pt]{article}

\usepackage{amsmath}
\usepackage{amssymb}
\usepackage[usenames,dvipsnames]{color}
\usepackage{comment}
\usepackage[utf8]{inputenc}
\usepackage{listings}
\usepackage{algorithm}
\usepackage{algpseudocode}
\usepackage{graphicx}
\usepackage{epstopdf}
\usepackage[top=3cm, bottom=3cm, left=2cm, right=2cm]{geometry}
\usepackage{caption}
\usepackage{color}
\usepackage[lofdepth,lotdepth]{subfig}
\usepackage[pdfborder={0 0 0}]{hyperref}
\usepackage[parfill]{parskip}
\usepackage{booktabs}

\definecolor{OliveGreen}{cmyk}{0.64,0,0.95,0.40}
\definecolor{Gray}{gray}{0.5}

\lstset{
    language=Java,
    basicstyle=\ttfamily,
    keywordstyle=\color{OliveGreen},
    commentstyle=\color{Gray},
    captionpos=b,
    breaklines=true,
    breakatwhitespace=false,
    showspaces=false,
    showtabs=false,
    numbers=left,
}

\title{VU Advanced Multiprocessor Programming \\
       SS 2013 \\
       Task 2: Concurrent Cuckoo Hash Set}
\author{Jakob Gruber, 0203440 \\
        Martin Kalany, 0825673}

\begin{document}

\maketitle

\section{Introduction}

Our task was to implement the \emph{Concurrent Cuckoo Hash Set} as presented
in \cite{herlihy}, pages 316-325. 

We benchmarked and compared two versions: The \emph{Striped Cuckoo Hash Set}
uses a fixed 2-by-L array of reentrant locks, where \lstinline|lock[i][j]|
guards the element at \lstinline|table[i][k]| with $k \mod L = j$. This variant
has the obvious disadvantage that the array of locks does not grow with the
number of elements $n$ stored in the set. Thus, we expect the performance to
decrease significantly for large $n$ and a small number of locks.

The \emph{Refined Cuckoo Hash Set} avoids this issue by letting the number of
locks grow with the the size of the set. This requires some additional
synchronization, the performance impact of which will be studied in Section
\ref{sec:performance}.

\section{Cuckoo Hash Set}

The foundation of of the presented data structure is \emph{Cuckoo Hashing},
an \textit{open-addressed} sequential hash set. This means that at each
location determined by a hash function, only one element can reside at any
given time\footnote{Or, as in this case, a fixed number of elements.}. In
contrast, a \emph{closed-addressed} hash set provides a bucket able to hold
multiple items at each location.

Cuckoo hashing uses two tables with different hash functions
\lstinline|h0(), h1()|. If \lstinline|h0(x)| maps a given element \lstinline|x|
to a location that is not occupied, \lstinline|x| is simply put there. If this
location happens to be already occupied, the currently stored element
\lstinline|y| is kicked out to make room for \lstinline|x|. \lstinline|y| is
relocated to \lstinline|h1(y)| using the same principle. This may of course
create long chains of relocations with circular dependencies. Thus, relocations
are limited to a fixed number of iterations before the capacity of the set and
thus the two tables is increased, prompting all elements to be re-added to the
set.  Figure \ref{alg:cuckooHashing} shows this algorithm.

The sets as implemented during this project use a slight variation of this
basic procedure. Each table address points to a data structure called probe sets
which can store a fixed number of items (\lstinline|PROBE_SIZE|), but prefer
to store \lstinline|PROBE_THRESHOLD| items or less. Relocation is triggered
by exceeding the \lstinline|PROBE_THRESHOLD| limit. Again, the set is resized
if no space is found for a particular item after a fixed number of relocation
iterations.

To allow concurrent access to the set, the \emph{Striped Cuckoo Hash Set}
adds a 2-by-L array of locks, with each protecting one or more probe sets.
This array of locks does not grow along with the set size, and thus turns into
a performance bottleneck as the number of threads and probe sets grow.

This limitation is avoided in the \emph{Refined Cuckoo Hash Set} by letting
the size lock array always equal the size of the corresponding array of probe
sets. Special care must be taken when acquiring locks while
\lstinline|resize()| is executing concurrently.

\begin{algorithm}
\caption{Cuckoo Hashing}
\label{alg:cuckooHashing}
\begin{algorithmic}[5]
\Function {put}{T x}
	\If{contains(x)}
		\State \Return false
	\EndIf
	\For{i = 0; i $<$ LIMIT; i++}
		\If {(x = swap(0, h0(x), x) == NULL}
			\State \Return true
		\ElsIf {(x = swap(1, h1(x), x) == NULL}
			\State \Return true
		\EndIf
	\EndFor
	\State resize()
	\State put(x)
\EndFunction
\end{algorithmic}
\end{algorithm}

\section{Implementation Details}
\label{sec:implDetails}

The Java code as presented in \cite{herlihy} posed several challenges for
a C++ implementation since it relies heavily on garbage collection (which C++
does not have). We were able to solve most of these and circumvented the rest
as follows:

\begin{itemize}
\item In \lstinline|relocate()|, a probe set is accessed without the
    protection of a lock. This causes two problems:
    
    On the one hand, this probe set might be accessed concurrently
    by another thread. This implies that all probe set functions must be
    thread-safe. Furthermore, the following code must be able to handle a
    situation in which all elements of the set have been removed before the
    thread running \lstinline|relocate| is able to call
    \lstinline|ProbeSet::get()|\footnote{The code in the book does not do
    this.}
    
    On the other hand, the probe set tables might be deleted by a concurrent
    call to \lstinline|resize()|. In Java, this is not an issue because an
    object is only garbage-collected when no more valid references to it exist.
    In C++ however, deletion is immediate and can result in segfaults if other
    threads dereference a pointer to the deleted object. 
    
    Both of these problems were avoided by requiring all locks to be held during
    a call to \lstinline|relocate()|. This (somewhat drastic) measure had to be
    taken for the lack of better alternatives; smart pointers would have been
    an option, but introduce a bottleneck whenever a probe set is accessed.
    Testing also showed that this global lock did not have a large effect on runtimes.

\item Locking in the \emph{Refined Cuckoo Hash Set} again relies on garbage collection, and
    in fact cannot be easily implemented without it. Specifically, there is no point
    in \lstinline|resize()| at which the locks may be deleted without possibly leaving
    other threads with invalid pointers. Keeping the locks in a reference-counted pointer
    solves this issue, but results in a single bottleneck affecting every method of the 
    hash set.
    
    We implemented the following solution: locks are never deleted, but simply added
    to a delete list which is processed in the hash set destructor. This results in
    some memory overhead, but preserves high performance without leaking memory.
\end{itemize}

\section{Progress Guarantees}
\label{sec:progressGuarantees}

We show that each public method of the class \lstinline|CuckooSet| in its
refined variant is deadlock-free\footnote{Unfortunately, \lstinline|acquire()|
is not starvation-free. We think this could be improved with some further
work.}.

Note that it is assumed that \lstinline|std::set| is
starvation-free if at no time the same element is accessed by more than one
thread. We ensure this by holding a lock for the item whenever it is accessed
to ensure mutual exclusion.

\begin{itemize}
\item For methods \lstinline|size()| and \lstinline|is_empty()| this is
    trivially true, since they only require an atomic load operation and no
    synchronisation whatsoever. They are thus starvation-free.
\end{itemize}

For the rest of the public methods, we need to have a closer look at methods of
other classes first:

\begin{itemize}
\item In class \lstinline|CuckooLock|, all methods are deadlock-free.
    Synchronization is handled by locking and unlocking mutexes. Since locks
    are accessed in the same order at all points (and the used
    \lstinline|std::recursive_lock| implementation is reentrant), deadlock may
    not occur. It is assumed that the callers ensure a correct sequence of
    calls to methods \lstinline|lock()| and \lstinline|unlock()| and that no
    thread stalls between calling \lstinline|lock()| and \lstinline|unlock()|. 

    Since the standard for C++ \cite{cppstandard} does not address the issue of
    fairness or starvation (relevant sections: 30.2.5 "Requirements for Lockable
    types" and 30.4 "Mutual exclusion"), we can not claim starvation-freedom for
    methods in class \lstinline|CuckooLock|. This is of little consequence for this
    paper, as we will show when analysing method \lstinline|acquire()|.

\item All methods in class \lstinline|AtomicMarkableReference| are trivially
    starvation-free since they consist solely of basic arithmetic and check-and-set
    operations.
\end{itemize}

In class \lstinline|CuckooSet|:

\begin{itemize}
\item In \lstinline|acquire()|, the caller loops until no \textit{other} thread
    has marked the set for resizing. Additionally, the caller will continue
    looping if another thread marked the set for resizing or the array of locks
    was recreated while the caller attempted to acquire the lock. This implies
    that the method is not starvation-free. It is deadlock-free: since we need
    to acquire only one lock and do not hold any other shared resources while
    attempting to do so. Note that this is sufficient for the method to not be
    starvation-free. If \lstinline|std::mutex| would  give any guarantees about
    fairness, it would not have an influence here. 

\item \lstinline|release()| is deadlock-free, since we only unlock a mutex.
\end{itemize}

Note that class \lstinline|GlobalLockGuard| is only a RAII-wrapper, which
acquires all locks in it's constructor by using \lstinline|CuckooLock|. It
further provides a method to release all locks and does so automatically upon
destruction. Thus, it's methods are deadlock-free. \lstinline|LockGuard| is
analogous, except for handling one lock for a specific item only. 

\begin{itemize}
\item \lstinline|contains()| simply acquires the necessary lock and is
    deadlock- free if the constructor and destructor of class
    \lstinline|LockGuard| are (which we already proved).

\item \lstinline|remove()| obtains the required lock and then calls
    \lstinline|contains()|, which again acquires the same lock. Since we use
    recursive mutexes, this is deadlock-free. 

\item Trusting that \lstinline|contains()| and \lstinline|remove()| are
    deadlock-free and assuming that \lstinline|put()| is as well (which will be
    proved immediately), it can be shown that \lstinline|relocate()| is too:
    The maximum number of iterations of the loop is limited by a constant and
    we only call starvation-free methods. Thus this method is deadlock-free.

\item \lstinline|resize()| and is similar to the above, except that the loop is
    bounded by a variable finite value.

\item \lstinline|put()| is trivially deadlock-free if \lstinline|resize()| and
    \lstinline|relocate()| are, since we have a simple sequence of statements. 
\end{itemize}

\section{Linearizability}
\label{sec:linearizability}

Most public functions in the cuckoo set variants consist of a single critical
section which is protected by calls to \lstinline|acquire()| and
\lstinline|release()|.  The linearization point of these functions is simply
the point at which \lstinline|acquire()| returns.

Both \lstinline|size()| and the derived function \lstinline|is_empty()| can be
linearized when the atomic load of \lstinline|the_size| completes.

\lstinline|put()| can contain three or more critical sections, each of which
may change parts of the set which other threads can access. Since there can not
be any single point at which the effects of the method seems to take place
instantaneously, \lstinline|put()| has no linearization point. Only be
restricting ourselves to calls which do not trigger either relocation or set
growth can we again place the linearization point at the conclusion of
\lstinline|acquire()|.

\section{Performance Analysis}
\label{sec:performance}

Benchmarking was performed on the saturn system. Pheet was compiled using
\verb|g++-4.7| and its default makeflags. The cuckoo set used the same
compiler as well as makeflags \verb|-Wall -Wextra -pedantic -std=gnu++11 -O3|.

Benchmarks were assembled by selecting the minimum execution time of 10 runs
for each value\footnote{With the exception of the \emph{Striped
Cuckoo Set} with only 8 locks, which exhibited very long execution times}.

Figures \ref{fig:plot1} and \ref{fig:plot2} show the performance for the
default set implementation provided by pheet using a global lock, our
\emph{Striped Cuckoo Set} implementation with 2x1024 locks and our
\emph{Refined Cuckoo Set}, both with an initial capacity of 2x1024 elements
and a limit of 512 relocate operations.  Figure \ref{fig:plot1} shows the
performance when 99\% of all operations on the set are calls to
\lstinline|contains()| and 0.5\% are calls to \lstinline|add()|. Figure
\ref{fig:plot2} shows the same for a benchmark where 50\% of all operations on
the set are calls to \lstinline|contains()| and 25\% are calls to
\lstinline|add()|.

Note that the x-axis is scaled with the binary logarithm in all diagrams. The
y-axis in Figure \ref{fig:plot3} is scaled with the decadic logarithm.

\begin{figure}[H]
\begin{center}
\includegraphics{099contains_005add.eps}
\end{center}
\caption{Performance comparison of different set implementations: 99\% contains
    and 0.5\% add operations}
\label{fig:plot1}
\end{figure}

\begin{figure}[H]
\begin{center}
\includegraphics{05contains_025add.eps}
\end{center}
\caption{Performance comparison of different set implementations: 50\% contains
    and 25\% add operations}
\label{fig:plot2}
\end{figure}

We expected both the striped and refined hash set variants to exhibit some
overhead in contrast to the global lock set for low thread counts. With rising
thread counts, the striped variant should scale very well up until a certain
point at which the number of locks (which does not increase with the number of
probe sets) becomes a bottleneck. The refined variant is expected to perform
well for all multi-threaded usage.

Results show these expectations to be justified. For single-threaded operation,
the global lock set performs best by a factor of 4. For two processors, all
implementations are roughly equal; all higher thread counts clearly favor
our concurrent set versions.

The striped variant has a small runtime advantage at smaller thread counts but
is superceded by the refined version at around 8 threads. We believe that 
Figure \ref{fig:plot2}
exhibits a larger discrepancy between these variants because the set becomes
less efficient once the the number of probe sets exceeds the number of locks,
and this set of benchmarks contains many more \lstinline|add()| calls - consequently the set contains more elements.

Surprising for us was how much performance was affected by using
reference counting (through \lstinline|std::shared_ptr|) to hold our locks.
Even though the implementation of the pointer is lock-free, peak performance was
reached at only 6 threads and decreased with increasing thread counts.

\begin{figure}[H]
\begin{center}
\includegraphics{refined.eps}
\end{center}
\caption{Performance comparison: \emph{Refined Cuckoo Set} (final implementation) vs. \emph{Refined Cuckoo Set} (reference counting)}
\label{fig:plot4}
\end{figure}


However, the performance of the \emph{Striped Cuckoo Hash Set} strongly
depends on the fixed number of locks. Although this variant performs well
with an array of locks of size 2x1024, performance decreases drastically with
less locks. This is illustrated in Figure \ref{fig:plot3}. Providing a larger
number of locks, on the other hand, increases memory overhead for applications
that require the set to hold comparatively few elements.

\begin{figure}[H]
\begin{center}
\includegraphics{striped.eps}
\end{center}
\caption{Performance comparison: \emph{Striped Cuckoo Set} with different amount of locks}
\label{fig:plot3}
\end{figure}

\section{Addendum: Go}

The \verb|go| subfolder\footnote{This was done by Jakob Gruber only} contains an implementation of the refined cuckoo hash set
written in the Go language. A couple of main points shaped this implementation
and made an interesting contrast to C++:

\begin{itemize}
\item Go has garbage collection. Therefore, most of the problems detailed in
    section \ref{sec:implDetails} were not an issue in Go.
\item Go does \emph{not} (yet) have generics. We have not found a way to
    decently implement containers without the use of generics\footnote{Of
    course, this doesn't mean that such a way does not exist.}. Go's built-in
    array, slice, and map containers all require compiler support.
\item The standard library has no reentrant lock (and discourages its use).
    We modified all functions to work with standard locks by passing the
    lock object into the function itself. If a lock had already been acquired,
    we simply passed a NOP lock.
\item It is not possible to retrieve a goroutine id (similar to a thread id in
    C++) in Go. Luckily, the modifications made in the previous point make
    goroutine ids redundant as well.
\end{itemize}

Further conveniences included the \lstinline|defer| statement (which executes
functions at the end of the current function) and integrated unit-test and
benchmark support. Unfortunately, we weren't able to use goroutines in this
project.

The following table shows the results of a single benchmarking run on a
quad core Intel i5 760 @ 2.80 GHz using Go version 1.1.1.

\begin{figure}[H]
\centering
\begin{tabular}{lll}
\toprule
Threads & \lstinline|add()| count & Runtime (ns/op) \\
\midrule
1  &   2000000 &            4958 ns/op \\
2  &   5000000 &            3172 ns/op \\
4  &   5000000 &            2723 ns/op \\
8  &   5000000 &            2759 ns/op \\
16 &   5000000 &            2723 ns/op \\
\bottomrule
\end{tabular}
\caption{Runtimes in nanoseconds per add() operation on 4 cores.}
\label{fig:go}
\end{figure}

\begin{thebibliography}{9}
\bibitem{herlihy}
    M. Herlihy, N. Shavit:
   \emph{The Art of Multiprocessor Programming}
\bibitem{cppstandard}	
	\emph{Working Draft, Standard for Programming Language C++},
	document number N3337, obtained on July $8^{th}$, 2013.	
    \url{http://www.open-std.org/jtc1/sc22/wg21/docs/papers/2012/n3337.pdf}
	\footnote{The standard is not available for free and not provided in the
    university's  library. The quoted draft is the first published after the
    C++11 standard, which it contains.}

\end{thebibliography}

\end{document}
